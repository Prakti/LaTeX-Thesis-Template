\chapter{Conclusions}
\label{chap:conclusions}
You generally cover three things in the Conclusions section, and each of these
usually merits a separate subsection:

\begin{enumerate}
    \item Conclusions
    \item Summary of Contributions
    \item Future Research
\end{enumerate}

Conclusions are not a rambling summary of the thesis: they are short, concise
statements of the inferences that you have made because of your work. It helps
to organize these as short numbered paragraphs, ordered from most to least
important. All conclusions should be directly related to the research question
stated in Section 4. Examples:

\begin{enumerate}
    \item The problem stated in Section 4 has been solved: as shown in Sections ? to
    ??, an algorithm capable of handling large-scale Zylon problems in reasonable
    time has been developed.
    \item The principal mechanism needed in the improved Zylon algorithm is the Grooty mechanism.
    \item Etc.
\end{enumerate}

The Summary of Contributions will be much sought and carefully read by the
examiners. Here you list the contributions of new knowledge that your thesis
makes. Of course, the thesis itself must substantiate any claims made here.
There is often some overlap with the Conclusions, but that's okay. Concise
numbered paragraphs are again best. Organize from most to least important.
Examples:

\begin{enumerate}
    \item Developed a much quicker algorithm for large-scale Zylon problems.
    \item Demonstrated the first use of the Grooty mechanism for Zylon calculations.
    \item Etc.
\end{enumerate}

The Future Research subsection is included so that researchers picking up this
work in future have the benefit of the ideas that you generated while you were
working on the project. Again, concise numbered paragraphs are usually best. 

% vim: set ft=tex
