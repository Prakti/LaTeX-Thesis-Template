\chapter{Review of the State of the Art}
\label{chap:relatedwork}
Here you review the state of the art relevant to your thesis. Again, a
different title is probably appropriate; e.g., "State of the Art in Zylon
Algorithms." The idea is to present (critical analysis comes a little bit
later) the major ideas in the state of the art right up to, but not including,
your own personal brilliant ideas.

You organize this section by idea, and not by author or by publication. For
example if there have been three important main approaches to Zylon Algorithms
to date, you might organize subsections around these three approaches, if
necessary:

\section{Iterative Approximation of Zylons}
\blindtext 

\section{Statistical Weighting of Zylons}
\blindtext

\section{Graph-Theoretic Approaches to Zylon Manipulation}
\blindtext

\section{A note from the Template Author}
If you are writing a master thesis, you can alternatively call this chapter
simply \textbf{'Related Work'}. It's also a good idea to try to come up with
more significant titles for your chapter than just these generic ones. If
unsure, ask your advisor how he/she likes it best.

\section{Some Filler Text to show a Sidewaysfigure}
This section shows off a custom-macro for placing figures sidweways, in case
they are to large and redesigning them is infeasible. As an example figure
\ref{fig:figure-label1} shows a large architecture diagram. 

\Blindtext

\SFIG[LOF caption for the sidewaysfigure]{The slightly longer caption,
containing additional information for the sidewaysfigure itself.}%
{figure-label1}{htbp}{%
            %\documentclass{article}
%\usepackage[a4paper, landscape, scale=0.7]{geometry}
%\usepackage[T1]{fontenc}
%\usepackage{mathptmx}
%\usepackage[scaled=0.9]{helvet}
%\usepackage{ascii}
%\usepackage{eulervm}
%\usepackage{amsmath}
%
%\usepackage{tikz}
%\usetikzlibrary{positioning}
\usetikzlibrary{automata}
\usetikzlibrary{arrows}
\usetikzlibrary{matrix}
\usetikzlibrary{fit}
\usetikzlibrary{calc}
\usetikzlibrary{chains}
\usetikzlibrary{patterns}
\usetikzlibrary{shadows}
\usetikzlibrary{shapes.geometric}
\usetikzlibrary{shapes.symbols}
\usetikzlibrary{shapes.arrows}
\usetikzlibrary{shapes.callouts}
\usetikzlibrary{decorations.shapes}
\usetikzlibrary{decorations.pathreplacing}
\usetikzlibrary{decorations.pathmorphing}

%%%% Define TikZ styles
%% Generic styles for pictures, nodes, ...
\tikzstyle{every picture}=[>=latex', double distance=2pt]
\tikzstyle{every node}=[font=\sffamily\small]

%% Utilities
\tikzstyle{reset}=[inner sep=1ex, minimum width=0mm, anchor=center]

%% Macros for MLA components
\tikzstyle{square}=[minimum width=8mm, minimum height=8mm]
\tikzstyle{component}=[draw, rectangle, rounded corners]
\tikzstyle{layer}=[component, drop shadow, fill=white]

%%%%% Setting up some layers
\pgfdeclarelayer{background}
\pgfdeclarelayer{foreground}
\pgfsetlayers{background,main,foreground}


%
%\begin{document}
%\centering
\begin{tikzpicture}
\tikzstyle{every node}=[font=\sffamily\footnotesize]

\node[layer, minimum width=9cm] (Core) { TinyOS Radio Core };

\node[layer, matrix, above=of Core, 
      column sep=5mm, row sep=30mm, inner sep=5mm,
      minimum width=9cm, nodes={reset, component}] (MacC) {
  \node (SenderC) {XmacSenderC}; & & \node (ReceiverC) {XmacReceiverC}; \\
  & \node (ChannelPollerC) {ChannelPollerC}; & \\
};

\node[layer, above=40mm of ChannelPollerC] (MacControlC) { MacControlC };

\node[layer, matrix, row sep=5mm, inner sep=2mm, nodes={reset, component}, 
  left=15mm of SenderC] (SenderDetail) {
  \node (XmacSenderP) { XmacSenderP }; \\
  \node (PreambleSenderC) { PreambleSenderC }; \\
};

\node[layer, matrix, row sep=5mm, inner sep=2mm, nodes={reset, component}, 
  right=10mm of ReceiverC] (ReceiverDetail) {
  \node (LplListenerC) { FixedSleepLplListenerC }; \\
  \node (RecvFilterP) { XmacReceiverFilterP }; \\
};


\coordinate[above=15mm of MacControlC] (Inputs) {};
\draw ([xshift=0mm] SenderC.south) coordinate (sout1) {};
\draw ([xshift=-5mm] SenderC.south) coordinate (sout2) {};

\draw ([xshift=0mm] ReceiverC.south) coordinate (rout1) {};
\draw ([xshift=5mm] ReceiverC.south) coordinate (rout2) {};

\coordinate[above=15mm of XmacSenderP] (xsin) {};
\coordinate[above=15mm of LplListenerC] (fslin) {};

\draw ([xshift=5mm] XmacSenderP.south) coordinate (xsout1) {};
\draw ([xshift=-5mm] XmacSenderP.south) coordinate (xsout2) {};
\draw ([xshift=10mm] XmacSenderP.east) coordinate (xsout3) {};

\draw ([xshift=8mm] LplListenerC.south) coordinate (fslout1) {};
\draw (LplListenerC.south) coordinate (fslout2) {};
\draw ([xshift=-8mm] LplListenerC.south) coordinate (fslout3) {};

\coordinate[below=27mm of PreambleSenderC] (psout) {};
\coordinate[below=27mm of RecvFilterP] (rcvout) {};

\draw ([xshift=-5mm] PreambleSenderC.south) coordinate (psout1) {};
\draw ([xshift=5mm] PreambleSenderC.south) coordinate (psout2) {};

\draw (RecvFilterP.south) coordinate (rcvout2) {};
\draw ([xshift=-8mm] RecvFilterP.south) coordinate (rcvout3) {};
\draw ([xshift=8mm] RecvFilterP.south) coordinate (rcvout1) {};


\draw[->] (Inputs) -- (MacControlC) node [midway, fill=white] { LowPowerListening };
\draw[->] (MacControlC) -- (MacControlC |- ChannelPollerC.north);
\draw[->] (SenderC) -| ([xshift=-5mm] ChannelPollerC.north);
%\draw[->] (ReceiverC) -| ([xshift=5mm] ChannelPollerC.north);
\draw[->] ([xshift=-5mm] ReceiverC.south) |- (ChannelPollerC.east) 
           node [pos=0.29, fill=white, rotate=90] { ChannelPoller };
\node[fill=white] at ([yshift=-25mm] MacControlC.south) (LPL) { LowPowerListening };
\draw[->] (ChannelPollerC.south) -- (MacControlC |- Core.north) 
           node [pos=0.65, fill=white] { ChannelMonitor };

\draw[->] (Inputs -| SenderC) -- (SenderC) node [midway, fill=white] { AsyncSend };
\draw[->] (sout1) -- (sout1 |- Core.north) node [pos=0.35, fill=white, rotate=90] { RadioPwrCtrl };
\draw[->] (sout2) -- (sout2 |- Core.north) node [pos=0.35, fill=white, rotate=90] { AsyncSend };

\draw[->] (xsin) -- (XmacSenderP) node [midway, fill=white] { AsyncSend };
\draw[->] (xsout1) -- (xsout1 |- PreambleSenderC.north);
\draw[->] (xsout2) -- (xsout2 |- PreambleSenderC.north);
\draw[->] (psout1) -- (psout1 |- psout) node [midway, fill=white, rotate=90] { AsyncSend };
\draw[->] (psout2) -- (psout2 |- psout) node [midway, fill=white, rotate=90] { RadioPwrCtrl };
\draw[->] (XmacSenderP.east) -- (xsout3) -- (xsout3 |- psout)
          node[pos=0.55, fill=white, rotate=90] { LowPowerListening };


\draw[->] (Inputs -| ReceiverC) -- (ReceiverC) node [midway, fill=white] { AsyncReceive };
\draw[->] (rout1) -- (rout1 |- Core.north) node [pos=0.35, fill=white, rotate=90] { RadioPwrCtrl };
\draw[->] (rout2) -- (rout2 |- Core.north) node [pos=0.35, fill=white, rotate=90] { AsyncReceive };

\draw[->] (fslin) -- (LplListenerC) node [midway, fill=white] { AsyncReceive };
\draw[->] (fslout1) -- (fslout1 |- RecvFilterP.north);
\draw[->] (fslout2) -- (fslout2 |- RecvFilterP.north);
\draw[->] (fslout3) -- (fslout3 |- RecvFilterP.north);
\draw[->] (rcvout1) -- (rcvout1 |- rcvout) node [midway, fill=white, rotate=90] { AsyncReceive };
\draw[->] (rcvout2) -- (rcvout2 |- rcvout) node [midway, fill=white, rotate=90] { RadioPwrCtrl };
\draw[->] (rcvout3) -- (rcvout3 |- rcvout) node [midway, fill=white, rotate=90] { ChannelPoller };

\draw[->] (Inputs -| ReceiverC) -- (ReceiverC) node [midway, fill=white] { AsyncReceive };

\draw[dashed] (SenderC.north west) -- (SenderDetail.north east);
\draw[dashed] (SenderC.south west) -- (SenderDetail.south east);

\draw[dashed] (ReceiverC.north east) -- (ReceiverDetail.north west);
\draw[dashed] (ReceiverC.south east) -- (ReceiverDetail.south west);

\end{tikzpicture}
%\end{document}

}

\Blindtext

% vim: set ft=tex
