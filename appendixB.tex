\chapter{Further remarks by Prof. Chinneck}
\label{app:b}
This appendix contains some remarks which followed the skeletal outline in the
original article by Prof. Chinneck. I needed to replace several things to make
them work with this template.

\section{Comments on the Thesis Skeleton}
Again, the thesis is a formal document designed to address the examiner's two
main questions. Chapter \ref{chap:relatedwork} and \ref{chap:problemstatement}
show that you have chosen a good problem, and Chapters
\ref{chap:contribution1} and \ref{chap:contribution2} show that you solved it.
Chapters \ref{chap:introduction} and \ref{chap:background} lead the reader
into the problem, and Chapter \ref{chap:conclusions} highlights the main
knowledge generated by the whole exercise.

Note also that everything that others did is carefully separated from
everything that you did. Knowing who did what is important to the examiners.
Section 4, the problem statement, is the obvious dividing line. That's the
main reason for putting it in the middle in this formal document.

\section{Getting Started}
The best way to get started on your thesis is to prepare an extended outline.
You begin by making up the Table of Contents, listing each section and
subsection that you propose to include. For each section and subsection, write
a brief point-form description of the contents of that section. The entire
outline might be 2 to 5 pages long. Now you and your thesis supervisor should
carefully review this outline: is there unnecessary material (i.e. not
directly related to the problem statement)? Then remove. Is there missing
material? Then add. It is much less painful and more time-efficient to make
such decisions early, during the outline phase, rather than after you've
already done a lot of writing which has to be thrown away.  


\section{How Long Does it Take to Write a Thesis?}
Longer than you think. Even after the research itself is all done -- models
built, calculations complete -- it is wise to allow at least one complete term
for writing the thesis. It's not the physical act of typing that takes so
long, it's the fact that writing the thesis requires the complete organization
of your arguments and results. It's during this formalization of your results
into a well-organized thesis document capable of withstanding the scrutiny of
expert examiners that you discover weaknesses. It's fixing those weaknesses
that takes time.

This is also probably the first time that your supervisor has seen the formal
expression of concepts that may have been approved previously in an informal
manner. Now is when you discover any misunderstandings or shortcomings in the
informal agreements. It takes time to fix these. Students for whom english is
not the mother tongue may have difficulty in getting ideas across, so that
numerous revisions are required. And, truth be known, supervisors are
sometimes not quick at reviewing and returning drafts.

\subsection*{Bottom line:} leave yourself enough time. A rush job has painful consequences
at the defence\footnote{Template Author's note: in case of a master thesis
this is not a defence but only a 'final presentation'. In any way, you need to
present your work and you might get awkward questions, based on the quality of
your work.}

\section{Tips}

Always keep the reader's backgrounds in mind. Who is your audience? How much
can you reasonably expect them to know about the subject before picking up
your thesis? Usually they are pretty knowledgeable about the general problem,
but they haven't been intimately involved with the details over the last
couple of years like you have: spell difficult new concepts out clearly. It
sometimes helps to mentally picture a real person that you know who has the
appropriate background, and to imagine that you are explaining your ideas
directly to that person.

Don't make the readers work too hard! This is fundamentally important. You
know what few questions the examiners need answers for (see above). Choose
section titles and wordings to clearly give them this information. The harder
they have to work to ferret out your problem, your defence of the problem,
your answer to the problem, your conclusions and contributions, the worse mood
they will be in, and the more likely that your thesis will need major
revisions.

A corollary of the above: it's impossible to be too clear! Spell things out
carefully, highlight important parts by appropriate titles etc. There's a huge
amount of information in a thesis: make sure you direct the readers to the
answers to the important questions.

Remember that a thesis is not a story: it usually doesn't follow the
chronology of things that you tried. It's a formal document designed to answer
only a few major questions.

Avoid using phrases like "Clearly, this is the case..." or "Obviously, it
follows that ..."; these imply that, if the readers don't understand, then
they must be stupid. They might not have understood because you explained it
poorly.

Avoid red flags, claims (like "software is the most important part of a
computer system") that are really only your personal opinion and not
substantiated by the literature or the solution you have presented. Examiners
like to pick on sentences like that and ask questions like, "Can you
demonstrate that software is the most important part of a computer system?"

\section{A Note on Computer Programs and Other Prototypes}
The purpose of your thesis is to clearly document an original contribution to
knowledge. You may develop computer programs, prototypes, or other tools as a
means of proving your points, but remember, the thesis is not about the tool,
it is about the contribution to knowledge. Tools such as computer programs are
fine and useful products, but you can't get an advanced degree just for the
tool. You must use the tool to demonstrate that you have made an original
contribution to knowledge; e.g., through its use, or ideas it embodies.

\section{Master's vs. PhD Thesis}
There are different expectations for Master's theses and for Doctoral theses.
This difference is not in format but in the significance and level of
discovery as evidenced by the problem to be solved and the summary of
contributions; a Doctoral thesis necessarily requires a more difficult problem
to be solved, and consequently more substantial contributions.

The contribution to knowledge of a Master's thesis can be in the nature of an
incremental improvement in an area of knowledge, or the application of known
techniques in a new area. The Ph.D. must be a substantial and innovative
contribution to knowledge.  

% vim: set ft=tex
