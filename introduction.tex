\chapter{Introduction}
\label{chap:introduction}
This note describes how to organize the written thesis which is the central
element of your graduate degree. To know how to organize the thesis document,
you first have to understand what graduate-level research is all about, so
that is covered too. In other words, this note should be helpful when you are
just getting started in your graduate program, as well as later when you start
to write your thesis.

\subsection{What Graduate Research is All About}
The distinguishing mark of graduate research is an original contribution to
knowledge. The thesis is a formal document whose sole purpose is to prove that
you have made an original contribution to knowledge. Failure to prove that you
have made such a contribution generally leads to failure.

To this end, your thesis must show two important things:

\begin{itemize}
    \item you have identified a worthwhile problem or question which has not been previously answered,
    \item you have solved the problem or answered the question.
\end{itemize}

Your contribution to knowledge generally lies in your solution or answer.
What the Graduate Thesis is All About

Because the purpose of the graduate thesis is to prove that you have made an
original and useful contribution to knowledge, the examiners read your thesis
to find the answers to the following questions:

\begin{itemize}
    \item What is this student's research question?
    \item Is it a good question? (has it been answered before? is it a useful question to work on?)
    \item Did the student convince me that the question was adequately answered?
    \item Has the student made an adequate contribution to knowledge?
\end{itemize}

A very clear statement of the question is essential to proving that you have
made an original and worthwhile contribution to knowledge. To prove the
originality and value of your contribution, you must present a thorough review
of the existing literature on the subject, and on closely related subjects.
Then, by making direct reference to your literature review, you must
demonstrate that your question (a) has not been previously answered, and (b)
is worth answering. Describing how you answered the question is usually easier
to write about, since you have been intimately involved in the details over
the course of your graduate work.

If your thesis does not provide adequate answers to the few questions listed
above, you will likely be faced with a requirement for major revisions or you
may fail your thesis defence outright. For this reason, the generic thesis
skeleton given below is designed to highlight the answers to those questions
with appropriate thesis organization and section titles. The generic thesis
skeleton can be used for any thesis. While some professors may prefer a
different organization, the essential elements in any thesis will be the same.
Some further notes follow the skeleton.

Always remember that a thesis is a formal document: every item must be in the
appropriate place, and repetition of material in different places should be
eliminated. 

\subsection{A note from the template-author}
This template is a skeleton, created after an article of Prof. John W.
Chinneck \cite{chinneck-thesis-howto}. Each following chapter will contain a short note on what Prof.
Chinneck thinks the chapter should contain. Another article/presentation worth
reading is this article by Michael A. Covington\cite{covington-write-think-learn}.
See the accompanied \texttt{thesis.bib} bibtex file for URL's to these
articles.

\subsection{What the introduction should contain}
This is a general introduction to what the thesis is all about -- it is not
just a description of the contents of each section. Briefly summarize the
question (you will be stating the question in detail later), some of the
reasons why it is a worthwhile question, and perhaps give an overview of your
main results. This is a birds-eye view of the answers to the main questions
answered in the thesis (see above).

\subsection{Some Blind Text as a filler}
\blindtext
\blindtext

% vim: set ft=tex
