\chapter{Your Contribution}
\label{chap:contribution1}
This part of the thesis is much more free-form. It may have one or
several sections and subsections. But it all has only one purpose: to convince
the examiners that you answered the question or solved the problem that you
set for yourself in Section 4. So show what you did that is relevant to
answering the question or solving the problem: if there were blind alleys and
dead ends, do not include these, unless specifically relevant to the
demonstration that you answered the thesis question. 

\section{Filler Text}
\Blindtext

\section{A note on acornyms}
This template uses a \LaTeX package for managing acronyms. Before usage, all
acronyms should be entered into the database contained in
\texttt{acronyms.tex} and can then be used. Example for acronyms are \ac{WSN},
\ac{CNF} or Like you can see from appendix \ref{chap:acronyms} Only the used
acronyms are shown in the final document.  \ac{IEEE}. If you want to used
acronym commands within section-titles, use the \verb|\acs| or \verb|\acl|
commands. You can manually reset the the arconyms marked as 'introduced', via
the \verb|\acresetall| command. This command is already called after including
the abstract into the main document (see \texttt{masterarbeit.tex}).



% vim: set ft=tex
