\chapter{Research Question or Problem Statement}
\label{chap:problemstatement}
Engineering theses tend to refer to a "problem" to be solved where other
disciplines talk in terms of a "question" to be answered. In either case, this
section has three main parts:

\begin{enumerate}
    \item A concise statement of the question that your thesis tackles
    \item Justification, by direct reference to section 3, that your question is previously unanswered
    \item Discussion of why it is worthwhile to answer this question.
\end{enumerate}

Item 2 above is where you analyze the information which you presented in
Section 3. For example, maybe your problem is to "develop a Zylon algorithm
capable of handling very large scale problems in reasonable time" (you would
further describe what you mean by "large scale" and "reasonable time" in the
problem statement). Now in your analysis of the state of the art you would
show how each class of current approaches fails (i.e. can handle only small
problems, or takes too much time). In the last part of this section you would
explain why having a large-scale fast Zylon algorithm is useful; e.g., by
describing applications where it can be used.

Since this is one of the sections that the readers are definitely looking for,
highlight it by using the word "problem" or "question" in the title: e.g.
"Research Question" or "Problem Statement", or maybe something more specific
such as "The Large-Scale Zylon Algorithm Problem." 

\section{A Note from the Template Author}
Again, this chapter might not be feasible for the scope of a master thesis.
Discuss with your advisor, if it is necessary, or if it might be better
integrated as the intoductory section of your contributions or as the last
section in the related work.

\section{Filler Text}
\blindtext
\blindtext
\blindenumerate
\blindtext
\blinddescription
\blindtext

% vim: set ft=tex
